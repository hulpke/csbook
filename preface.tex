\chapter{Preface}

For many decades, Calculus has been the epitome of College level mathematics.
This is due to its undoubted usefulness in modelling the physical world for
applications in Engineering and in the Physical Sciences.

But this apex role, and the associated standardization of the subject as seen
in the manifold Calculus textbooks on the market, have led to a number of
drawbacks:
\begin{itemize}
\item The underlying assumption of the course in topics and examples is
that everyone ultimately wants to solve differential equations.
\item Univariate calculus typically is spread over two semesters, delaying
the point for students to take Linear Algebra.
\item 
Material and presentation are firmly rooted in the 19th century, bypassing
much of the developments that underly modern mathematics and its applications
in information technology.
\item The course (in particular as far as student learning is concerned) is
heavily reliant on being able to execute recipe methods for finding
(anti-)derivatives, tasks that nowadays are more than satisfactory solved by
computer programs.
\item 
With standard problems and scores providing an easy grade distribution, and
asa a class with a significant failure rate, Calculus is often abused as a
filter, standing in as a test for study skills and {\em grit}.
\end{itemize}
All of this makes the standard Calculus course an awkward choice for majors
in disciplines that focus on data analysis and information processing --
disciplines that barely existed when the standard Calculus sequence was
created.

This book therefore takes a new approach to an introductory College
mathematics course for students in Computational Science disciplines: It
starts with mathematical foundations, sufficient to enable students to take
more advanced courses such as Linear Algebra, Combinatorics, Elementary
Number Theory, or Abstract Algebra, which are highly relevant to their major.
It then presents 

The focus here is on functions as data, and what Calculus tools can do
conceptionally, rather than on modelling physical phenomena or practicing
manipulation of functions given through term expressions. Nor shall we delve
into the borderline cases of the definitions -- such as functions that are
once but not twice differentiable, discontinuities for the sake of being
discontinuous, or the behavior of series on the circle of convergence.

This does not mean that we will be superficial. We shall cover the concepts
from univariate calculus, with applications to, and as relevant to,
Computational Science, but we do not focus on solving the ``standard''
problems one will find on a typical Calculus exam.




\section*{Thanks}


I am grateful to Kirk Bonney, Harley Meade, Michael Moy, Tristan Neighbors
and  Rachel Tremaine, graduate students at CSU, who corrected mistakes,
provided examples and applications, as well as alternative text for graphics.
Their work was supported by an Open Educational Materials grant from the
Colorado Department of Higher Education, administered through the CSU
Libraries, which is gratefully acknowledged.
