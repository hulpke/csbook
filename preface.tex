\chapter{Preface}

For many decades, Calculus has been the epitome of College level
mathematics and the $800$lb gorilla --- make that a $800/\varepsilon$ lb
gorilla ---  of challenging STEM classes.
This is due to its undoubted usefulness in modeling the physical world for
applications in Engineering and in the Physical Sciences. 

But this apex role, and the associated standardization of the subject as seen
in the manifold Calculus textbooks on the market, have led to a number of
drawbacks:
\begin{itemize}
\item The underlying assumption of the course in topics and examples is
that everyone ultimately wants to solve differential equations.
\item Univariate calculus typically is spread over two semesters, delaying
the point for students to take Linear Algebra.
\item 
Material and presentation are firmly rooted in the 19th century, bypassing
much of the developments that underly modern mathematics and its applications
in information technology.
\item The course (in particular as far as student learning is concerned) is
heavily reliant on being able to execute recipe methods for finding
(anti-)derivatives, tasks that nowadays are more than satisfactory solved by
computer programs.
\item 
With standard problems and scores providing an easy grade distribution, and
as a a class with a significant failure rate, Calculus is often abused as a
filter, standing in as a test for study skills and {\em grit}.
\end{itemize}
All of this makes the standard Calculus course an awkward choice for majors
in disciplines that focus on data analysis and information processing --
disciplines that barely existed when todays standard Calculus sequence was
created. The huge amount of material that is at best peripherally relevant,
\medskip

This book therefore takes a new approach to an introductory College
mathematics course for students in Computational Science disciplines: It
starts with mathematical foundations, sufficient to enable students to take
subsequently
more advanced courses such as Linear Algebra, Combinatorics, Elementary
Number Theory, or Abstract Algebra, which are highly relevant to their major.

Overall, the focus is on functions as data, and what Calculus tools can do
conceptionally, rather than on modeling physical phenomena or practicing
manipulation of functions given through term expressions. Nor shall we delve
into the borderline cases of the definitions -- such as functions that are
once but not twice differentiable, discontinuities for the sake of being
discontinuous, or the behavior of series on the circle of convergence.

This does not mean that we will be superficial. We cover most concepts
from univariate calculus, with applications to, and as relevant to,
Computational Science, but we do not focus on solving the ``standard''
problems one will find on a typical Calculus exam.
\medskip

Compared with a ``classical'' Calculus course, the concept of discontinuous
or non-differential functions is deemphasized -- data is discrete and as
such always differentiable. Nor is the rote training of calculating
(anti)derivatives touched upon -- appropriately for computational
applications we present symbolic differentiation as an algorithm.  Limits
occur mainly in the context of asymptotic growth classes as needed for
computational complexity.  Taylor polynomials introduce the concept of
approximation and error estimations for desired accuracy. They also serve to
justify statements about the derivatives of elementary functions and to
introduce relevant manipulations of power series.

The fundamental theorem of calculus, and basic integration techniques are
introduced, but the underlying assumption is that students will never
have to calculate an antiderivative by hand.
\medskip

This reflects my experience as a professional mathematician in an
area\mynote{Abstract Algebra, concretely Computational Group Theory}
close to Theoretical Computer Science: Since the third year of College I
have, outside teaching a Calculus class myself, not
encountered the need to calculate a symbolic (anti-)derivative, prove
convergence of a series, or work around a one-sided limit.
\smallskip

Forcing students to master these techniques, that they will never need
professionally, is a disservice. While progress in the 19th and 20th Century 
was ultimately shaped through classical Engineering,  the 21st Century has
shown itself to be built on understanding and manipulating data. This book
presents Calculus fit for this time.

On the other hand, it
needs to be clear that disciplines built on describing the physical world.
through differential equations might find the version of Calculus presented
here as too weak. 
\medskip

Still, I can imagine, and indeed would welcome, if this version of Calculus
would be of use for other disciplines, and I hope that students and
instructors will find it a useful addition to the existing market of
Calculus books.

\section*{Thanks}

I am immensely grateful to Kirk Bonney, Harley Meade, Michael Moy, Tristan Neighbors
and  Rachel Tremaine, graduate students in the mathematics department at
Colorado State University, who corrected mistakes,
provided examples and applications, as well as alternative text for graphics.

Their work was supported by an Open Educational Materials grant from the
Colorado Department of Higher Education, administered through the CSU
Libraries, which is gratefully acknowledged.

\bigskip

Fort Collins, CO, in January 2022,

Alexander Hulpke
